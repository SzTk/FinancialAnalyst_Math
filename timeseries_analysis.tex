\documentclass[dvipdfmx,autodetect-engine, unicode, 10pt, aspectratio=169]{beamer}

% \documentclass[dvipdfmx,autodetect-engine]{jsarticle}
% \usepackage{luatexja}% 日本語
% \usepackage[haranoaji,deluxe]{luatexja-preset}% フォント指定
\renewcommand{\kanjifamilydefault}{\gtdefault}% 既定をゴシック体に

\usetheme[progressbar=frametitle]{metropolis}
\usepackage{appendixnumberbeamer}
\usepackage{booktabs}
\usepackage[scale=2]{ccicons}
\usepackage{pgfplots}
\usepgfplotslibrary{dateplot}
\usepackage{xspace}
\newcommand{\themename}{\textbf{\textsc{metropolis}}\xspace}
\usepackage{adjustbox}
\usepackage{caption}
\captionsetup[figure]{font=tiny}
\usepackage{fancyvrb} % verbatim replacement that allows latex
\usepackage{listings} % Setting for Code block
\usepackage{bm} % for Bold font in Math \bm

\title{時系列解析}
\subtitle{証券アナリスト}
% \date{\today}
\date{}
\author{Takayuki Suzuki}
\institute{This is institude of the author}

\begin{document}

\maketitle


\begin{frame}{時系列解析目標}
    \begin{itemize}
        \item トレンドの推定
        \item 定常性
        \item 自己回帰モデルで推定する
        \item 単位根仮定、ランダムウォークを理解
        \item 構造変化、季節調整
        \item ボラティリティ変動モデル
        \item 非定常時系列の見せかけの回帰と共和分
        \item 応用する
    \end{itemize}
\end{frame}

\begin{frame}{トレンドモデル}
    トレンドモデル
\end{frame}
\begin{frame}{対数線形トレンドモデル}
    対数モデル
\end{frame}
\begin{frame}{自己共分散}
    自己共分散とコレログラム
\end{frame}
\begin{frame}{定常性}
    定常性
\end{frame}
\begin{frame}{ARモデル}
    自己回帰モデル
\end{frame}
\begin{frame}{ARモデルでの予想}
    予想
\end{frame}
\begin{frame}{非定常字時系列}
    非定常とは、
\end{frame}
\begin{frame}{単位根検定}
    単位根
\end{frame}
\begin{frame}{ランダムウォーク}
    ランダムウォーク
\end{frame}
\begin{frame}{季節性}
    季節性
\end{frame}
\begin{frame}{共和分}
    共和分とは
\end{frame}
\end{document}
