\documentclass[dvipdfmx,autodetect-engine, unicode, 10pt, aspectratio=169]{beamer}

% \documentclass[dvipdfmx,autodetect-engine]{jsarticle}
% \usepackage{luatexja}% 日本語
% \usepackage[haranoaji,deluxe]{luatexja-preset}% フォント指定
\renewcommand{\kanjifamilydefault}{\gtdefault}% 既定をゴシック体に

\usetheme[progressbar=frametitle]{metropolis}
\usepackage{appendixnumberbeamer}
\usepackage{booktabs}
\usepackage[scale=2]{ccicons}
\usepackage{pgfplots}
\usepgfplotslibrary{dateplot}
\usepackage{xspace}
\newcommand{\themename}{\textbf{\textsc{metropolis}}\xspace}
\usepackage{adjustbox}
\usepackage{fancyvrb} % verbatim replacement that allows latex
\usepackage{listings} % Setting for Code block
\usepackage{bm} % for Bold font in Math \bm

\title{ポートフォリオ分析と線形代数}
\subtitle{証券アナリスト}
% \date{\today}
\date{}
\author{Takayuki Suzuki}
\institute{This is institude of the author}

\begin{document}

\maketitle


\begin{frame}{線形代数目標}
    線形代数(ベクトルと行列)では、「理解する」と「計算できる」が有る。 
    \begin{itemize}
        \item 計算できるようにするもの
        \begin{itemize}
            \item 2x2の固有方程式から固有ベクトルが計算できる
            \item 2x2の対角化
            \item 関数を成分とするベクトルの微分表記と関数の計算ができる
        \end{itemize}
        \item 理解すればいいもの
        \begin{itemize}
            \item 基本的計算
            \item 内積の応用
            \item 数理モデルの行列での表記
            \item 行列式や逆行列について
            \item 期待値、分散、共分散のベクトルの二次形式
        \end{itemize}
    \end{itemize}
    計算できる必要があるのは、2x2の固有方程式から固有ベクトルを計算すること \\
    および対角化、そして、関数を成分とするベクトルの微分表記、計算ができること。 \\
    
\end{frame}

\begin{frame}{理解しておく用語のリスト}
    % \begin{minipage}[b]{0.3\linewidth}
    % hogehoge
    % \end{minipage}
    % \begin{minipage}[b]{0.3\linewidth}
    % fugafuga
    % \end{minipage}
    \begin{minipage}[t]{0.45\linewidth}
    \begin{itemize}
        \item 内積
        \item 直交
        \item 転置(転置行列)
        \item 正方行列
        \item 対角成分、非対角成分
        \item トレース
        \item 単位行列
        \item 対称行列
        \item ゼロベクトル、ゼロ行列
    \end{itemize}
    \end{minipage}
    \begin{minipage}[t]{0.45\linewidth}
    \begin{itemize}
        \item 行列式
        \item 逆行列
        \item 正則行列
        \item 固有値, 固有ベクトル
        \item 直交行列
        \item 対角化とスペクトル分解
        \item 一次形式と二次形式
        \item 一次形式と二次形式それぞれの微分
    \end{itemize}
    \end{minipage}
\end{frame}

\begin{frame}{計算できるようにしておくもの}
    計算をできるようにしておくこと
    \begin{itemize}
        \item 2x2行列の固有ベクトルの算出
        \item 2x2行列の対角化
    \end{itemize}
\end{frame}

\begin{frame}{2x2行列の固有ベクトルの算出}
    行列$\bm{A}$に対して、スカラー$\lambda$とベクトル$\bm{x}$が
    \begin{align*}
        \bm{A}\bm{x} = \lambda \bm{x}
    \end{align*}
    という関係を満たすとき、\\
    $\lambda$を行列$\bm{A}$の固有値(Eigenvalue), $\bm{x}$を行列$\bm{A}$の固有ベクトル(Eigenvector)という。\\
    行列Aの固有値を求めろ、という計算問題は出題される可能性がある。\\
    \vspace{5pt}

    固有値と固有ベクトルは、
    \begin{align*}
        \text{det}(\bm{A} - \lambda \bm{I})
    \end{align*}
    がゼロである条件から求める。$\text{det}$は行列式、$\bm{I}$は単位行列である。 \\
    この式を特性方程式(または固有方程式)とよぶ。
\end{frame}


\begin{frame}{固有値と固有ベクトルの求め方例}
    例;\\
    $\bm{A} =
        \begin{bmatrix}
            1 && 2 \\
            3 && 2
        \end{bmatrix}
    $の固有値を求める。\\
    $
    \bm{A} - \lambda \bm{I} = 
    \begin{bmatrix}
        1-\lambda && 2 \\
        3 && 2 - \lambda 
    \end{bmatrix}
    $
    であるので、その行列式は、
    \begin{align*}
        \text{det}(\bm{A} - \lambda \bm{x}) = (1-\lambda)(2-\lambda) - 2\cdot 3
    \end{align*}
    である。これがゼロである条件を計算する。  
    \begin{align*}
        (1-\lambda)(2-\lambda) - 2\cdot3 = \lambda^2 - 3\lambda - 4 = 0
    \end{align*}
    より、$\lambda = -1, 4$が固有値である。\\
    \vspace{5pt}
\scriptsize $\begin{bmatrix}
        a && b \\
        c && d
    \end{bmatrix}$の行列式が、$(ad-bc)$であったことを覚えておく
\end{frame}

\begin{frame}{固有値と固有ベクトルの求め方例}
    つぎに、それぞれの固有値に対して、固有ベクトルを求める。
    固有値が$\lambda = -1$のとき、固有値の定義から、
    \begin{align*}
        (\bm{A} - (-1)\cdot \bm{I} )\bm{x} = 0
    \end{align*}
    であるから、
    \begin{align*}
        &(\bm{A} - \bm{I})\bm{x} = 
            \begin{bmatrix}
                1-(-1) && 2 \\
                3 && 2 - (-1)
            \end{bmatrix}
            \begin{bmatrix}
                x_1 \\ x_2
            \end{bmatrix} = 
            \begin{bmatrix}
                2 && 2 \\
                3 && 3
            \end{bmatrix}
            \begin{bmatrix}
                x_1 \\ x_2
            \end{bmatrix} = 0 \\
            &\Rightarrow 2x_1 + 2x_2 = 0 \text{ and } 3x_1 + 3x_2 = 0
    \end{align*}
    これを満たすベクトルは無数にあるが、長さが1である単位ベクトルを選ぶと、\\
    $x = \frac{1}{\sqrt{2}}\begin{bmatrix}
        1 \\ -1
    \end{bmatrix}$
    が対応する固有ベクトルである。\\もう一方の固有値$\lambda =4$に対する固有ベクトルも、同様に求める。答えは
    $\frac{1}{\sqrt{13}}\begin{bmatrix}
        2 \\ 3
    \end{bmatrix}$である。
    
\end{frame}

\begin{frame}{対角化}
    固有ベクトルを並べた行列Pで元の行列を挟むと、結果、対角成分だけになる性質がある。
    2x2行列で示す。\\
    $\bm{A}$の固有値を$\lambda_1, \lambda_2$, 固有ベクトルを
    $\bm{x}_1=\begin{bmatrix}
        x_{1,1} \\ x_{1,2}
    \end{bmatrix}, \bm{x}_2=\begin{bmatrix}
        x_{2,1} \\ x_{2,2}
    \end{bmatrix}$
    とする。固有ベクトルを並べた行列Pを考える。
    \begin{align*}
        \bm{P} = \begin{bmatrix}
            \bm{x}_1&&\bm{x}_2
        \end{bmatrix} = \begin{bmatrix}
            x_{1,1} && x_{2,1} \\
            x_{1,2} && x_{2,2}
        \end{bmatrix}
    \end{align*}
    元の行列$\bm{A}$にこの行列$\bm{P}$とその逆行列$\bm{P}^{-1}$を左右から掛けると
    \begin{align*}
        \bm{P}^{-1}\bm{A}\bm{P} = \begin{bmatrix}
            \lambda_1 && 0 \\
            0 && \lambda_2
        \end{bmatrix}
    \end{align*}
    となる。(計算すると、必ずそうなる)\\
    この操作を、行列の対角化という。\\
    さらに、対角化の式を逆転させると、元の行列Aが、固有ベクトルの直積の和で表現できる。
    これを、行列のスペクトル分解という。\footnotesize (スペクトル分解の話は、たぶん無理なのでやめておく)
\end{frame}

\begin{frame}{ポートフォリオマネージメントへの応用}
    \begin{center}
        \Large ポートフォリオマネージメントへの応用    
    \end{center}
\end{frame}
\begin{frame}[fragile]{分散共分散行列}
    通常の分散は、確率変数に対して計算される。つまり、\\
    $\text{var}(x)$と書いた場合、xは確率変数であった。つまり単純な定数ではなく、値が一定の確率分布によってランダムに定まる変数であった。 
    \footnote{xが定数であれば、$\text{E}(x)=x, \text{var}(x)=0$であるが、xが確率変数であれば、$\text{E}(x)=\bar{x}, \text{var}(x)=\sigma^2_x$である。}
    \\
    では、ベクトル変数$\bm{x}$に対する分散はどうなるか? 分散共分散行列になる。\\
    例えば、3つの資産のリターン(これは確率変数である)から成る下記のベクトルを考える。
    \[\bm{r} = (r_1, r_2, r_3)\]
    これの分散は、実は下記のような行列になり、これを分散共分散行列という。(覚える)
    \begin{align*}
        \text{var}(\bm{r}) = 
        \begin{bmatrix}
        \text{var}(r_1) & \text{cov}(r_1, r_2) & \text{cov}(r_1, r_3)\\
        \text{cov}(r_2, r_1) & \text{var}(r_2) & \text{cov}(r_2, r_3)\\
        \text{cov}(r_3, r_1) & \text{cov}(r_3, r_2) & \text{var}(r_3)
        \end{bmatrix}
    \end{align*}
    
\end{frame}

\begin{frame}{一次形式とその微分}
    一次形式とは、ベクトルの世界の一次式で、係数ベクトル$\bm{a}$と、変数ベクトル$\bm{x}$の積で表される式のこと。
    \begin{align*}
        f(\bm{x}) = \bm{a}^T\bm{x}
    \end{align*}
    これのxでの微分すると下記になる。
    \begin{align*}
        \frac{\partial f(\bm{x})}{\partial \bm{x}} = \bm{a}
    \end{align*}
    つまり、結果的には普通の一次式の微分と同じになる。
\end{frame}

\begin{frame}{二次形式とその微分}
    二次形式とは、ベクトルの世界の二次式である。\\
    この場合、係数はもはやベクトルではなく、行列になる。\\
    すなわち、係数行列を$\bm{A}$とすると、二次形式は、
    \begin{align*}
        f(\bm{x}) = \bm{x}^{T}\bm{A}\bm{x}
    \end{align*}
    これを微分すると、下記になる。
    \begin{align*}
        \frac{\partial f(\bm{x})}{\partial \bm{x}} = 2\bm{A} \bm{x}
    \end{align*}
    つまり、結果的には、普通の二次式の微分とほぼ、同じになる。\\
    これらの微分のやり方は、結果だけ知っておけばいいはず。\\
    分散共分散行列と、一次形式、二次形式の微分は、次の効用関数最適化への応用で利用される。
\end{frame}

\begin{frame}{効用関数最適化への応用}
    $x$を各資産の配分比率としたとき、効用関数Uが
    \begin{align*}
        &\text{U} = \bm{x}^T \bm{\mu} - \frac{1}{2\tau}\bm{x}^T\Sigma\bm{x} \\\
        &\text{\scriptsize ただし$\mu$は各資産の期待リターンベクトル、$\Sigma$は各資産の分散共分散行列}
    \end{align*}
    で表せた場合の、最適比率を求める。最適の時、微分がゼロになることから、
    \begin{align*}
        \frac{\partial \text{U}}{\partial \bm{x}} = 0
    \end{align*}
    の条件から求めることができればよい。
    \begin{align*}
        \frac{\partial \text{U}}{\partial \bm{x}} = \bm{\mu} - \frac{1}{\tau} \Sigma \bm{x}
    \end{align*}
    であるから、これがゼロになる条件を見つければよい。
    \begin{align*}
        &\bm{\mu} - (1/\tau) \Sigma \bm{x} = 0 \\
        &\Rightarrow \bm{x} =  \tau\Sigma^{-1}\bm{\mu}
    \end{align*}
    となり、$\tau$、$\Sigma$, $\bm{\mu}$が分かっていれば、上記を計算することで最適分配比率を求めることができる。やることは、行列とベクトルの掛け算である。
\end{frame}

\begin{frame}{トービンの分離定理}
    結果だけ覚えておくこと。\\
    前記のような効用関数の場合、上記の最適投資比率の計算式から言える。リスク許容度$\tau$はスカラーであり、すべての成分に対して一律同じ値がかかるためである。
    \begin{enumerate}
        \item リスク資産間の投資比率は、リスク許容度$\tau$には依存しない。
        \item リスク許容度$\tau$によって変わるのは、リスク資産と無リスク資産の比のみである。
    \end{enumerate}
    リスク許容度の上昇は、リスク資産への投資を増加させ、無リスク資産への投資を減少させる。リスク許容度が変わっても、リスク資産の中での分配比率は変わらない。    \\
    リスク資産内での分配比率が変化するのは、期待リターン$\mu$が変わったり、分散共分散(つまり各資産の期待リスクや、資産間の共分散)$\Sigma$が変わった場合である。
\end{frame}

\end{document}
